% Options for packages loaded elsewhere
\PassOptionsToPackage{unicode}{hyperref}
\PassOptionsToPackage{hyphens}{url}
\PassOptionsToPackage{dvipsnames,svgnames*,x11names*}{xcolor}
%
\documentclass[
]{article}
\usepackage{lmodern}
\usepackage{amssymb,amsmath}
\usepackage{ifxetex,ifluatex}
\ifnum 0\ifxetex 1\fi\ifluatex 1\fi=0 % if pdftex
  \usepackage[T1]{fontenc}
  \usepackage[utf8]{inputenc}
  \usepackage{textcomp} % provide euro and other symbols
\else % if luatex or xetex
  \usepackage{unicode-math}
  \defaultfontfeatures{Scale=MatchLowercase}
  \defaultfontfeatures[\rmfamily]{Ligatures=TeX,Scale=1}
\fi
% Use upquote if available, for straight quotes in verbatim environments
\IfFileExists{upquote.sty}{\usepackage{upquote}}{}
\IfFileExists{microtype.sty}{% use microtype if available
  \usepackage[]{microtype}
  \UseMicrotypeSet[protrusion]{basicmath} % disable protrusion for tt fonts
}{}
\makeatletter
\@ifundefined{KOMAClassName}{% if non-KOMA class
  \IfFileExists{parskip.sty}{%
    \usepackage{parskip}
  }{% else
    \setlength{\parindent}{0pt}
    \setlength{\parskip}{6pt plus 2pt minus 1pt}}
}{% if KOMA class
  \KOMAoptions{parskip=half}}
\makeatother
\usepackage{xcolor}
\IfFileExists{xurl.sty}{\usepackage{xurl}}{} % add URL line breaks if available
\IfFileExists{bookmark.sty}{\usepackage{bookmark}}{\usepackage{hyperref}}
\hypersetup{
  pdftitle={Week 6 - Simulation Project},
  pdfauthor={Aman Arora},
  colorlinks=true,
  linkcolor=Maroon,
  filecolor=Maroon,
  citecolor=Blue,
  urlcolor=cyan,
  pdfcreator={LaTeX via pandoc}}
\urlstyle{same} % disable monospaced font for URLs
\usepackage[margin=1in]{geometry}
\usepackage{color}
\usepackage{fancyvrb}
\newcommand{\VerbBar}{|}
\newcommand{\VERB}{\Verb[commandchars=\\\{\}]}
\DefineVerbatimEnvironment{Highlighting}{Verbatim}{commandchars=\\\{\}}
% Add ',fontsize=\small' for more characters per line
\usepackage{framed}
\definecolor{shadecolor}{RGB}{248,248,248}
\newenvironment{Shaded}{\begin{snugshade}}{\end{snugshade}}
\newcommand{\AlertTok}[1]{\textcolor[rgb]{0.94,0.16,0.16}{#1}}
\newcommand{\AnnotationTok}[1]{\textcolor[rgb]{0.56,0.35,0.01}{\textbf{\textit{#1}}}}
\newcommand{\AttributeTok}[1]{\textcolor[rgb]{0.77,0.63,0.00}{#1}}
\newcommand{\BaseNTok}[1]{\textcolor[rgb]{0.00,0.00,0.81}{#1}}
\newcommand{\BuiltInTok}[1]{#1}
\newcommand{\CharTok}[1]{\textcolor[rgb]{0.31,0.60,0.02}{#1}}
\newcommand{\CommentTok}[1]{\textcolor[rgb]{0.56,0.35,0.01}{\textit{#1}}}
\newcommand{\CommentVarTok}[1]{\textcolor[rgb]{0.56,0.35,0.01}{\textbf{\textit{#1}}}}
\newcommand{\ConstantTok}[1]{\textcolor[rgb]{0.00,0.00,0.00}{#1}}
\newcommand{\ControlFlowTok}[1]{\textcolor[rgb]{0.13,0.29,0.53}{\textbf{#1}}}
\newcommand{\DataTypeTok}[1]{\textcolor[rgb]{0.13,0.29,0.53}{#1}}
\newcommand{\DecValTok}[1]{\textcolor[rgb]{0.00,0.00,0.81}{#1}}
\newcommand{\DocumentationTok}[1]{\textcolor[rgb]{0.56,0.35,0.01}{\textbf{\textit{#1}}}}
\newcommand{\ErrorTok}[1]{\textcolor[rgb]{0.64,0.00,0.00}{\textbf{#1}}}
\newcommand{\ExtensionTok}[1]{#1}
\newcommand{\FloatTok}[1]{\textcolor[rgb]{0.00,0.00,0.81}{#1}}
\newcommand{\FunctionTok}[1]{\textcolor[rgb]{0.00,0.00,0.00}{#1}}
\newcommand{\ImportTok}[1]{#1}
\newcommand{\InformationTok}[1]{\textcolor[rgb]{0.56,0.35,0.01}{\textbf{\textit{#1}}}}
\newcommand{\KeywordTok}[1]{\textcolor[rgb]{0.13,0.29,0.53}{\textbf{#1}}}
\newcommand{\NormalTok}[1]{#1}
\newcommand{\OperatorTok}[1]{\textcolor[rgb]{0.81,0.36,0.00}{\textbf{#1}}}
\newcommand{\OtherTok}[1]{\textcolor[rgb]{0.56,0.35,0.01}{#1}}
\newcommand{\PreprocessorTok}[1]{\textcolor[rgb]{0.56,0.35,0.01}{\textit{#1}}}
\newcommand{\RegionMarkerTok}[1]{#1}
\newcommand{\SpecialCharTok}[1]{\textcolor[rgb]{0.00,0.00,0.00}{#1}}
\newcommand{\SpecialStringTok}[1]{\textcolor[rgb]{0.31,0.60,0.02}{#1}}
\newcommand{\StringTok}[1]{\textcolor[rgb]{0.31,0.60,0.02}{#1}}
\newcommand{\VariableTok}[1]{\textcolor[rgb]{0.00,0.00,0.00}{#1}}
\newcommand{\VerbatimStringTok}[1]{\textcolor[rgb]{0.31,0.60,0.02}{#1}}
\newcommand{\WarningTok}[1]{\textcolor[rgb]{0.56,0.35,0.01}{\textbf{\textit{#1}}}}
\usepackage{graphicx,grffile}
\makeatletter
\def\maxwidth{\ifdim\Gin@nat@width>\linewidth\linewidth\else\Gin@nat@width\fi}
\def\maxheight{\ifdim\Gin@nat@height>\textheight\textheight\else\Gin@nat@height\fi}
\makeatother
% Scale images if necessary, so that they will not overflow the page
% margins by default, and it is still possible to overwrite the defaults
% using explicit options in \includegraphics[width, height, ...]{}
\setkeys{Gin}{width=\maxwidth,height=\maxheight,keepaspectratio}
% Set default figure placement to htbp
\makeatletter
\def\fps@figure{htbp}
\makeatother
\setlength{\emergencystretch}{3em} % prevent overfull lines
\providecommand{\tightlist}{%
  \setlength{\itemsep}{0pt}\setlength{\parskip}{0pt}}
\setcounter{secnumdepth}{-\maxdimen} % remove section numbering

\title{Week 6 - Simulation Project}
\author{Aman Arora}
\date{}

\begin{document}
\maketitle

\begin{center}\rule{0.5\linewidth}{0.5pt}\end{center}

\begin{itemize}
\tightlist
\item
  Simulation Study 1
\end{itemize}

** Introduction

\begin{Shaded}
\begin{Highlighting}[]
\NormalTok{birthday =}\StringTok{ }\DecValTok{12121977}
\KeywordTok{set.seed}\NormalTok{(birthday)}

\NormalTok{df1 =}\StringTok{ }\KeywordTok{read.csv}\NormalTok{(}\StringTok{"study_1.csv"}\NormalTok{)}
\CommentTok{#No of simulations}
\NormalTok{m =}\StringTok{ }\DecValTok{2000}

\CommentTok{#No of parameters}
\NormalTok{p =}\StringTok{ }\DecValTok{4}

\CommentTok{#Function to Simulate MLR}
\NormalTok{sim_mlr =}\StringTok{ }\ControlFlowTok{function}\NormalTok{(df, betas, sigma) \{}
\NormalTok{  n =}\StringTok{ }\KeywordTok{nrow}\NormalTok{(df)}
\NormalTok{  X =}\StringTok{ }\KeywordTok{cbind}\NormalTok{(}\DataTypeTok{x0=} \KeywordTok{rep}\NormalTok{(}\DecValTok{1}\NormalTok{,n), }\KeywordTok{as.matrix}\NormalTok{(df[}\OperatorTok{-}\DecValTok{1}\NormalTok{]))}
\NormalTok{  epsilon =}\StringTok{ }\KeywordTok{rnorm}\NormalTok{(n, }\DataTypeTok{mean =} \DecValTok{0}\NormalTok{, }\DataTypeTok{sd =}\NormalTok{ sigma)}
\NormalTok{  Y =}\StringTok{ }\NormalTok{X }\OperatorTok\StringTok{ }\NormalTok{betas1 }\OperatorTok{+}\StringTok{ }\NormalTok{epsilon}
\NormalTok{  df[}\StringTok{"y"}\NormalTok{] =}\StringTok{ }\NormalTok{Y}
  \KeywordTok{return}\NormalTok{(df)}
\NormalTok{\}}

\CommentTok{#sigmas to be simulated}
\NormalTok{sigmas =}\StringTok{ }\KeywordTok{c}\NormalTok{(}\DecValTok{1}\NormalTok{,}\DecValTok{5}\NormalTok{,}\DecValTok{10}\NormalTok{)}

\CommentTok{#Model 1}
\NormalTok{betas1 =}\StringTok{ }\KeywordTok{c}\NormalTok{(}\DecValTok{3}\NormalTok{,}\DecValTok{1}\NormalTok{,}\DecValTok{1}\NormalTok{,}\DecValTok{1}\NormalTok{)}

\CommentTok{#Initialize matrices to store results for model 1}
\NormalTok{f_stat_}\DecValTok{1}\NormalTok{ =}\StringTok{ }\KeywordTok{matrix}\NormalTok{(}\DecValTok{0}\NormalTok{, m, }\KeywordTok{length}\NormalTok{(sigmas))}
\NormalTok{p_val_}\DecValTok{1}\NormalTok{ =}\StringTok{  }\KeywordTok{matrix}\NormalTok{(}\DecValTok{0}\NormalTok{, m, }\KeywordTok{length}\NormalTok{(sigmas))}
\NormalTok{r_squared_}\DecValTok{1}\NormalTok{ =}\StringTok{ }\KeywordTok{matrix}\NormalTok{(}\DecValTok{0}\NormalTok{, m, }\KeywordTok{length}\NormalTok{(sigmas))}
\NormalTok{n =}\StringTok{ }\KeywordTok{nrow}\NormalTok{(df1)}

\CommentTok{#Simulate MLR for each sigma and store results}
\ControlFlowTok{for}\NormalTok{(j }\ControlFlowTok{in} \DecValTok{1}\OperatorTok{:}\KeywordTok{length}\NormalTok{(sigmas))}
\NormalTok{\{  }
  \ControlFlowTok{for}\NormalTok{ (i }\ControlFlowTok{in} \DecValTok{1}\OperatorTok{:}\NormalTok{m)}
\NormalTok{  \{}
\NormalTok{    df1 =}\StringTok{ }\KeywordTok{sim_mlr}\NormalTok{(df1,betas1,sigmas[j])}
\NormalTok{    model_fitted_sum =}\StringTok{ }\KeywordTok{summary}\NormalTok{(}\KeywordTok{lm}\NormalTok{(y }\OperatorTok{~}\StringTok{ }\NormalTok{x1}\OperatorTok{+}\NormalTok{x2}\OperatorTok{+}\NormalTok{x3, }\DataTypeTok{data =}\NormalTok{ df1) )}
\NormalTok{    f_stat_}\DecValTok{1}\NormalTok{[i,j] =}\StringTok{ }\NormalTok{model_fitted_sum}\OperatorTok{$}\NormalTok{fstatistic[}\StringTok{"value"}\NormalTok{]}
\NormalTok{    r_squared_}\DecValTok{1}\NormalTok{[i,j]=}\StringTok{ }\NormalTok{model_fitted_sum}\OperatorTok{$}\NormalTok{r.squared}
\NormalTok{    p_val_}\DecValTok{1}\NormalTok{[i,j] =}\StringTok{ }\KeywordTok{pf}\NormalTok{(f_stat_}\DecValTok{1}\NormalTok{[i,j],p}\DecValTok{-1}\NormalTok{,n}\OperatorTok{-}\NormalTok{p,}\DataTypeTok{lower.tail =} \OtherTok{FALSE}\NormalTok{)}
\NormalTok{  \}}
\NormalTok{\}  }
\end{Highlighting}
\end{Shaded}

\begin{Shaded}
\begin{Highlighting}[]
\CommentTok{#Model 2}

\NormalTok{betas1 =}\StringTok{ }\KeywordTok{c}\NormalTok{(}\DecValTok{3}\NormalTok{,}\DecValTok{0}\NormalTok{,}\DecValTok{0}\NormalTok{,}\DecValTok{0}\NormalTok{)}

\CommentTok{#Initialize matrices to store results for model 2}
\NormalTok{f_stat_}\DecValTok{2}\NormalTok{ =}\StringTok{ }\KeywordTok{matrix}\NormalTok{(}\DecValTok{0}\NormalTok{, m, }\KeywordTok{length}\NormalTok{(sigmas))}
\NormalTok{p_val_}\DecValTok{2}\NormalTok{ =}\StringTok{  }\KeywordTok{matrix}\NormalTok{(}\DecValTok{0}\NormalTok{, m, }\KeywordTok{length}\NormalTok{(sigmas))}
\NormalTok{r_squared_}\DecValTok{2}\NormalTok{ =}\StringTok{ }\KeywordTok{matrix}\NormalTok{(}\DecValTok{0}\NormalTok{, m, }\KeywordTok{length}\NormalTok{(sigmas))}

\CommentTok{#Simulate MLR for each sigma and store results}
\ControlFlowTok{for}\NormalTok{(j }\ControlFlowTok{in} \DecValTok{1}\OperatorTok{:}\KeywordTok{length}\NormalTok{(sigmas))}
\NormalTok{\{  }
  \ControlFlowTok{for}\NormalTok{ (i }\ControlFlowTok{in} \DecValTok{1}\OperatorTok{:}\NormalTok{m)}
\NormalTok{  \{}
\NormalTok{    df1 =}\StringTok{ }\KeywordTok{sim_mlr}\NormalTok{(df1,betas1,sigmas[j])}
\NormalTok{    model_fitted_sum =}\StringTok{ }\KeywordTok{summary}\NormalTok{(}\KeywordTok{lm}\NormalTok{(y }\OperatorTok{~}\StringTok{ }\NormalTok{x1}\OperatorTok{+}\NormalTok{x2}\OperatorTok{+}\NormalTok{x3, }\DataTypeTok{data =}\NormalTok{ df1) )}
\NormalTok{    f_stat_}\DecValTok{2}\NormalTok{[i,j] =}\StringTok{ }\NormalTok{model_fitted_sum}\OperatorTok{$}\NormalTok{fstatistic[}\StringTok{"value"}\NormalTok{]}
\NormalTok{    r_squared_}\DecValTok{2}\NormalTok{[i,j]=}\StringTok{ }\NormalTok{model_fitted_sum}\OperatorTok{$}\NormalTok{r.squared}
\NormalTok{    p_val_}\DecValTok{2}\NormalTok{[i,j] =}\StringTok{ }\KeywordTok{pf}\NormalTok{(f_stat_}\DecValTok{2}\NormalTok{[i,j],p}\DecValTok{-1}\NormalTok{,n}\OperatorTok{-}\NormalTok{p,}\DataTypeTok{lower.tail =} \OtherTok{FALSE}\NormalTok{)}
\NormalTok{  \}}
\NormalTok{\}  }
\end{Highlighting}
\end{Shaded}

\hypertarget{simulation-study-2-using-rmse-for-selection}{%
\section{Simulation Study 2: Using RMSE for
Selection?}\label{simulation-study-2-using-rmse-for-selection}}

In homework we saw how Test RMSE can be used to select the ``best''
model. In this simulation study we will investigate how well this
procedure works. Since splitting the data is random, we don't expect it
to work correctly each time. We could get unlucky. But averaged over
many attempts, we should expect it to select the appropriate model.

We will simulate from the model

\[
Y_i = \beta_0 + \beta_1 x_{i1} + \beta_2 x_{i2} + \beta_3 x_{i3} + \beta_4 x_{i4} + \beta_5 x_{i5} + \beta_6 x_{i6} + \epsilon_i
\]

where \(\epsilon_i \sim N(0, \sigma^2)\) and

\begin{itemize}
\tightlist
\item
  \(\beta_0 = 0\),
\item
  \(\beta_1 = 3\),
\item
  \(\beta_2 = -4\),
\item
  \(\beta_3 = 1.6\),
\item
  \(\beta_4 = -1.1\),
\item
  \(\beta_5 = 0.7\),
\item
  \(\beta_6 = 0.5\).
\end{itemize}

We will consider a sample size of \(500\) and three possible levels of
noise. That is, three values of \(\sigma\).

\begin{itemize}
\tightlist
\item
  \(n = 500\)
\item
  \(\sigma \in (1, 2, 4)\)
\end{itemize}

Use the data found in \href{study_2.csv}{\texttt{study\_2.csv}} for the
values of the predictors. These should be kept constant for the entirety
of this study. The \texttt{y} values in this data are a blank
placeholder.

Each time you simulate the data, randomly split the data into train and
test sets of equal sizes (250 observations for training, 250
observations for testing).

For each, fit \textbf{nine} models, with forms:

\begin{itemize}
\tightlist
\item
  \texttt{y\ \textasciitilde{}\ x1}
\item
  \texttt{y\ \textasciitilde{}\ x1\ +\ x2}
\item
  \texttt{y\ \textasciitilde{}\ x1\ +\ x2\ +\ x3}
\item
  \texttt{y\ \textasciitilde{}\ x1\ +\ x2\ +\ x3\ +\ x4}
\item
  \texttt{y\ \textasciitilde{}\ x1\ +\ x2\ +\ x3\ +\ x4\ +\ x5}
\item
  \texttt{y\ \textasciitilde{}\ x1\ +\ x2\ +\ x3\ +\ x4\ +\ x5\ +\ x6},
  the correct form of the model as noted above
\item
  \texttt{y\ \textasciitilde{}\ x1\ +\ x2\ +\ x3\ +\ x4\ +\ x5\ +\ x6\ +\ x7}
\item
  \texttt{y\ \textasciitilde{}\ x1\ +\ x2\ +\ x3\ +\ x4\ +\ x5\ +\ x6\ +\ x7\ +\ x8}
\item
  \texttt{y\ \textasciitilde{}\ x1\ +\ x2\ +\ x3\ +\ x4\ +\ x5\ +\ x6\ +\ x7\ +\ x8\ +\ x9}
\end{itemize}

For each model, calculate Train and Test RMSE.

\[
\text{RMSE}(\text{model, data}) = \sqrt{\frac{1}{n} \sum_{i = 1}^{n}(y_i - \hat{y}_i)^2}
\]

Repeat this process with \(1000\) simulations for each of the \(3\)
values of \(\sigma\). For each value of \(\sigma\), create a plot that
shows how average Train RMSE and average Test RMSE changes as a function
of model size. Also show the number of times the model of each size was
chosen for each value of \(\sigma\).

Done correctly, you will have simulated the \(y\) vector \(3×1000=3000\)
times. You will have fit \(9×3×1000=27000\) models. A minimal result
would use \(3\) plots. Additional plots may also be useful.

Potential discussions:

\begin{itemize}
\tightlist
\item
  Does the method \textbf{always} select the correct model? On average,
  does is select the correct model?
\item
  How does the level of noise affect the results?
\end{itemize}

\begin{Shaded}
\begin{Highlighting}[]
\NormalTok{df1 =}\StringTok{ }\KeywordTok{read.csv}\NormalTok{(}\StringTok{"study_2.csv"}\NormalTok{)}
\CommentTok{#No of simulations}
\NormalTok{m =}\StringTok{ }\DecValTok{500}
\KeywordTok{library}\NormalTok{(knitr)}
\KeywordTok{set.seed}\NormalTok{(}\DecValTok{12121977}\NormalTok{)}
\end{Highlighting}
\end{Shaded}

\begin{Shaded}
\begin{Highlighting}[]
\CommentTok{#Function to Simulate MLR}
\NormalTok{sim_mlr =}\StringTok{ }\ControlFlowTok{function}\NormalTok{(df, betas, sigma) \{}
\NormalTok{  n =}\StringTok{ }\KeywordTok{nrow}\NormalTok{(df)}
\NormalTok{  X =}\StringTok{ }\KeywordTok{cbind}\NormalTok{(}\DataTypeTok{x0=} \KeywordTok{rep}\NormalTok{(}\DecValTok{1}\NormalTok{,n), }\KeywordTok{as.matrix}\NormalTok{(df[}\DecValTok{2}\OperatorTok{:}\DecValTok{7}\NormalTok{]))}
\NormalTok{  epsilon =}\StringTok{ }\KeywordTok{rnorm}\NormalTok{(n, }\DataTypeTok{mean =} \DecValTok{0}\NormalTok{, }\DataTypeTok{sd =}\NormalTok{ sigma)}
\NormalTok{  Y =}\StringTok{ }\NormalTok{X }\OperatorTok\StringTok{ }\NormalTok{betas }\OperatorTok{+}\StringTok{ }\NormalTok{epsilon}
\NormalTok{  df[}\StringTok{"y"}\NormalTok{] =}\StringTok{ }\NormalTok{Y}
  \KeywordTok{return}\NormalTok{(df)}
\NormalTok{\}}

\CommentTok{#sigmas to be simulated}
\NormalTok{sigma =}\StringTok{ }\KeywordTok{c}\NormalTok{(}\DecValTok{1}\NormalTok{,}\DecValTok{2}\NormalTok{,}\DecValTok{4}\NormalTok{)}

\CommentTok{#Model betas}
\NormalTok{betas1 =}\StringTok{ }\KeywordTok{c}\NormalTok{(}\DecValTok{0}\NormalTok{,}\DecValTok{3}\NormalTok{,}\OperatorTok{-}\DecValTok{4}\NormalTok{,}\FloatTok{1.6}\NormalTok{,}\OperatorTok{-}\FloatTok{1.1}\NormalTok{,}\FloatTok{0.7}\NormalTok{,}\FloatTok{0.5}\NormalTok{)}
\end{Highlighting}
\end{Shaded}

\begin{Shaded}
\begin{Highlighting}[]
\CommentTok{#Function to calculate and return training RMSE and test RMSE for a model}
\NormalTok{evaluate_model <-}\ControlFlowTok{function}\NormalTok{(fitted_model,train_data, test_data)}
\NormalTok{\{}

\NormalTok{ rmse_train =}\StringTok{ }\KeywordTok{sqrt}\NormalTok{(}\KeywordTok{mean}\NormalTok{(fitted_model}\OperatorTok{$}\NormalTok{residuals}\OperatorTok{^}\DecValTok{2}\NormalTok{))}
  
\NormalTok{ new_data1 =}\StringTok{ }\NormalTok{test_data[,}\KeywordTok{names}\NormalTok{(fitted_model}\OperatorTok{$}\NormalTok{model)][}\OperatorTok{-}\DecValTok{1}\NormalTok{]}
 
\NormalTok{ y_test_pred =}\StringTok{ }\KeywordTok{predict}\NormalTok{(fitted_model, }\DataTypeTok{newdata =}\NormalTok{ new_data1) }

\NormalTok{ residual_test =}\StringTok{   }\KeywordTok{as.vector}\NormalTok{(y_test_pred }\OperatorTok{-}\StringTok{ }\KeywordTok{as.vector}\NormalTok{(test_data[}\StringTok{"y"}\NormalTok{]))[,}\DecValTok{1}\NormalTok{]}

\NormalTok{ rmse_test  =}\StringTok{ }\KeywordTok{sqrt}\NormalTok{(}\KeywordTok{mean}\NormalTok{(residual_test}\OperatorTok{^}\DecValTok{2}\NormalTok{))}
  
\NormalTok{ result =}\StringTok{ }\KeywordTok{c}\NormalTok{(rmse_train, rmse_test)}

\NormalTok{\}}
\end{Highlighting}
\end{Shaded}

\begin{Shaded}
\begin{Highlighting}[]
\NormalTok{result_test =}\KeywordTok{rep}\NormalTok{(}\DecValTok{0}\NormalTok{,}\DecValTok{1000}\NormalTok{)  }
\NormalTok{result_train =}\KeywordTok{rep}\NormalTok{(}\DecValTok{0}\NormalTok{,}\DecValTok{1000}\NormalTok{)  }

\KeywordTok{set.seed}\NormalTok{(}\DecValTok{12121977}\NormalTok{)}
\ControlFlowTok{for}\NormalTok{(q }\ControlFlowTok{in} \DecValTok{1}\OperatorTok{:}\DecValTok{1000}\NormalTok{)}
\NormalTok{\{}
\NormalTok{  df1 =}\StringTok{ }\KeywordTok{sim_mlr}\NormalTok{(df1,betas1,}\DecValTok{4}\NormalTok{)}
\NormalTok{  model_true =}\StringTok{ }\KeywordTok{lm}\NormalTok{(y}\OperatorTok{~}\NormalTok{x1}\OperatorTok{+}\NormalTok{x2}\OperatorTok{+}\NormalTok{x3}\OperatorTok{+}\NormalTok{x4}\OperatorTok{+}\NormalTok{x5}\OperatorTok{+}\NormalTok{x6,df1)}
  \CommentTok{#Split training and test data}
\NormalTok{  trn_idx =}\StringTok{ }\KeywordTok{sample}\NormalTok{(}\DecValTok{1}\OperatorTok{:}\KeywordTok{nrow}\NormalTok{(df1), }\DecValTok{250}\NormalTok{)}
\NormalTok{  tst_idx =}\StringTok{ }\KeywordTok{setdiff}\NormalTok{(}\DecValTok{1}\OperatorTok{:}\KeywordTok{nrow}\NormalTok{(df1), trn_idx)}
\NormalTok{  trn_data =}\StringTok{ }\NormalTok{df1[trn_idx,]}
\NormalTok{  tst_data  =}\StringTok{ }\NormalTok{df1[tst_idx,]}

\NormalTok{  result_train[q] =}\StringTok{ }\KeywordTok{evaluate_model}\NormalTok{(model_true,trn_data,tst_data)[}\DecValTok{1}\NormalTok{]}
\NormalTok{  result_test[q] =}\StringTok{ }\KeywordTok{evaluate_model}\NormalTok{(model_true,trn_data,tst_data)[}\DecValTok{2}\NormalTok{]}
\NormalTok{\}}

\KeywordTok{mean}\NormalTok{(result_train)}
\end{Highlighting}
\end{Shaded}

\begin{verbatim}
## [1] 3.971
\end{verbatim}

\begin{Shaded}
\begin{Highlighting}[]
\KeywordTok{mean}\NormalTok{(result_test)}
\end{Highlighting}
\end{Shaded}

\begin{verbatim}
## [1] 3.961
\end{verbatim}

\begin{Shaded}
\begin{Highlighting}[]
\NormalTok{result_test =}\KeywordTok{rep}\NormalTok{(}\DecValTok{0}\NormalTok{,}\DecValTok{1000}\NormalTok{)  }
\NormalTok{result_train =}\KeywordTok{rep}\NormalTok{(}\DecValTok{0}\NormalTok{,}\DecValTok{1000}\NormalTok{)  }
  
\KeywordTok{set.seed}\NormalTok{(}\DecValTok{12121977}\NormalTok{)}

\ControlFlowTok{for}\NormalTok{(q }\ControlFlowTok{in} \DecValTok{1}\OperatorTok{:}\DecValTok{1000}\NormalTok{)}
\NormalTok{\{}
\NormalTok{  df1 =}\StringTok{ }\KeywordTok{sim_mlr}\NormalTok{(df1,betas1,}\DecValTok{4}\NormalTok{)}
\NormalTok{  model_all =}\StringTok{ }\KeywordTok{lm}\NormalTok{(y }\OperatorTok{~}\StringTok{ }\NormalTok{x1}\OperatorTok{+}\NormalTok{x2}\OperatorTok{+}\NormalTok{x3}\OperatorTok{+}\NormalTok{x4}\OperatorTok{+}\NormalTok{x5}\OperatorTok{+}\NormalTok{x6}\OperatorTok{+}\NormalTok{x7}\OperatorTok{+}\NormalTok{x8}\OperatorTok{+}\NormalTok{x9 , }\DataTypeTok{data =}\NormalTok{ df1)}
  \CommentTok{#Split training and test data}
\NormalTok{  trn_idx =}\StringTok{ }\KeywordTok{sample}\NormalTok{(}\DecValTok{1}\OperatorTok{:}\KeywordTok{nrow}\NormalTok{(df1), }\DecValTok{250}\NormalTok{)}
\NormalTok{  tst_idx =}\StringTok{ }\KeywordTok{setdiff}\NormalTok{(}\DecValTok{1}\OperatorTok{:}\KeywordTok{nrow}\NormalTok{(df1), trn_idx)}
\NormalTok{  trn_data =}\StringTok{ }\NormalTok{df1[trn_idx,]}
\NormalTok{  tst_data  =}\StringTok{ }\NormalTok{df1[tst_idx,]}
\NormalTok{  result_train[q] =}\StringTok{ }\KeywordTok{evaluate_model}\NormalTok{(model_all,trn_data,tst_data)[}\DecValTok{1}\NormalTok{]}
\NormalTok{  result_test[q] =}\StringTok{ }\KeywordTok{evaluate_model}\NormalTok{(model_all,trn_data,tst_data)[}\DecValTok{2}\NormalTok{]}
\NormalTok{\}}

\KeywordTok{mean}\NormalTok{(result_train)}
\end{Highlighting}
\end{Shaded}

\begin{verbatim}
## [1] 3.959
\end{verbatim}

\begin{Shaded}
\begin{Highlighting}[]
\KeywordTok{mean}\NormalTok{(result_test)}
\end{Highlighting}
\end{Shaded}

\begin{verbatim}
## [1] 3.95
\end{verbatim}

\hypertarget{simulation-study-3-power}{%
\section{Simulation Study 3: Power}\label{simulation-study-3-power}}

In this simulation study we will investigate the \textbf{power} of the
significance of regression test for simple linear regression.

\[
H_0: \beta_{1} = 0 \ \text{vs} \ H_1: \beta_{1} \neq 0
\]

Recall, we had defined the \emph{significance} level, \(\alpha\), to be
the probability of a Type I error.

\[
\alpha = P[\text{Reject } H_0 \mid H_0 \text{ True}] = P[\text{Type I Error}]
\]

Similarly, the probability of a Type II error is often denoted using
\(\beta\); however, this should not be confused with a regression
parameter.

\[
\beta = P[\text{Fail to Reject } H_0 \mid H_1 \text{ True}] = P[\text{Type II Error}]
\]

\emph{Power} is the probability of rejecting the null hypothesis when
the null is not true, that is, the alternative is true and \(\beta_{1}\)
is non-zero.

\[
\text{Power} = 1 - \beta = P[\text{Reject } H_0 \mid H_1 \text{ True}]
\]

Essentially, power is the probability that a signal of a particular
strength will be detected. Many things affect the power of a test. In
this case, some of those are:

\begin{itemize}
\tightlist
\item
  Sample Size, \(n\)
\item
  Signal Strength, \(\beta_1\)
\item
  Noise Level, \(\sigma\)
\item
  Significance Level, \(\alpha\)
\end{itemize}

We'll investigate the first three.

To do so we will simulate from the model

\[
Y_i = \beta_0 + \beta_1 x_i + \epsilon_i
\]

where \(\epsilon_i \sim N(0, \sigma^2)\).

For simplicity, we will let \(\beta_0 = 0\), thus \(\beta_1\) is
essentially controlling the amount of ``signal.'' We will then consider
different signals, noises, and sample sizes:

\begin{itemize}
\tightlist
\item
  \(\beta_1 \in (-2, -1.9, -1.8, \ldots, -0.1, 0, 0.1, 0.2, 0.3, \ldots 1.9, 2)\)
\item
  \(\sigma \in (1, 2, 4)\)
\item
  \(n \in (10, 20, 30)\)
\end{itemize}

We will hold the significance level constant at \(\alpha = 0.05\).

Use the following code to generate the predictor values, \texttt{x}:
values for different sample sizes.

\begin{Shaded}
\begin{Highlighting}[]
\NormalTok{x_values =}\StringTok{ }\KeywordTok{seq}\NormalTok{(}\DecValTok{0}\NormalTok{, }\DecValTok{5}\NormalTok{, }\DataTypeTok{length =}\NormalTok{ n)}
\end{Highlighting}
\end{Shaded}

For each possible \(\beta_1\) and \(\sigma\) combination, simulate from
the true model at least \(1000\) times. Each time, perform the
significance of the regression test. To estimate the power with these
simulations, and some \(\alpha\), use

\[
\hat{\text{Power}} = \hat{P}[\text{Reject } H_0 \mid H_1 \text{ True}] = \frac{\text{# Tests Rejected}}{\text{# Simulations}}
\]

It is \emph{possible} to derive an expression for power mathematically,
but often this is difficult, so instead, we rely on simulation.

Create three plots, one for each value of \(\sigma\). Within each of
these plots, add a ``power curve'' for each value of \(n\) that shows
how power is affected by signal strength, \(\beta_1\).

Potential discussions:

\begin{itemize}
\tightlist
\item
  How do \(n\), \(\beta_1\), and \(\sigma\) affect power? Consider
  additional plots to demonstrate these effects.
\item
  Are \(1000\) simulations sufficient?
\end{itemize}

\begin{Shaded}
\begin{Highlighting}[]
\CommentTok{#Function to Simulate SLR}
\NormalTok{sim_slr =}\StringTok{ }\ControlFlowTok{function}\NormalTok{(x, beta_}\DecValTok{0}\NormalTok{ , beta_}\DecValTok{1}\NormalTok{, sigma) \{}
\NormalTok{  n =}\StringTok{ }\KeywordTok{length}\NormalTok{(x)}
\NormalTok{  epsilon =}\StringTok{ }\KeywordTok{rnorm}\NormalTok{(n, }\DataTypeTok{mean =} \DecValTok{0}\NormalTok{, }\DataTypeTok{sd =}\NormalTok{ sigma)}
\NormalTok{  y =}\StringTok{ }\NormalTok{beta_}\DecValTok{0} \OperatorTok{+}\StringTok{ }\NormalTok{beta_}\DecValTok{1} \OperatorTok{*}\StringTok{ }\NormalTok{x }\OperatorTok{+}\StringTok{ }\NormalTok{epsilon}
  \KeywordTok{data.frame}\NormalTok{(}\DataTypeTok{predictor =}\NormalTok{ x, }\DataTypeTok{response =}\NormalTok{ y)}
\NormalTok{\}}

\NormalTok{x_values =}\StringTok{ }\KeywordTok{seq}\NormalTok{(}\DecValTok{0}\NormalTok{, }\DecValTok{5}\NormalTok{, }\DataTypeTok{length =} \DecValTok{40}\NormalTok{)}

\NormalTok{beta1 =}\StringTok{ }\KeywordTok{seq}\NormalTok{(}\DataTypeTok{from =} \DecValTok{-2}\NormalTok{, }\DataTypeTok{to =} \DecValTok{2}\NormalTok{, }\DataTypeTok{by =} \FloatTok{.1}\NormalTok{)}

\NormalTok{sigma =}\StringTok{ }\KeywordTok{c}\NormalTok{(}\DecValTok{1}\NormalTok{,}\DecValTok{2}\NormalTok{,}\DecValTok{4}\NormalTok{)}

\NormalTok{n =}\StringTok{ }\KeywordTok{c}\NormalTok{(}\DecValTok{10}\NormalTok{,}\DecValTok{20}\NormalTok{,}\DecValTok{40}\NormalTok{)}

\CommentTok{#sim_slr(n,0,beta1,sigma[1])}
\end{Highlighting}
\end{Shaded}

\end{document}
